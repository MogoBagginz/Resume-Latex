%%%%%%%%%%%%%%%%%
% This is an sample CV template created using altacv.cls
% (v1.3, 10 May 2020) written by LianTze Lim (liantze@gmail.com). Now compiles with pdfLaTeX, XeLaTeX and LuaLaTeX.
% This fork/modified version has been made by Nicolás Omar González Passerino (nicolas.passerino@gmail.com, 15 Oct 2020)
%
%% It may be distributed and/or modified under the
%% conditions of the LaTeX Project Public License, either version 1.3
%% of this license or (at your option) any later version.
%% The latest version of this license is in
%%    http://www.latex-project.org/lppl.txt
%% and version 1.3 or later is part of all distributions of LaTeX
%% version 2003/12/01 or later.
%%%%%%%%%%%%%%%%

%% If you need to pass whatever options to xcolor
\PassOptionsToPackage{dvipsnames}{xcolor}

%% If you are using \orcid or academicons
%% icons, make sure you have the academicons
%% option here, and compile with XeLaTeX
%% or LuaLaTeX.
% \documentclass[10pt,a4paper,academicons]{altacv}

%% Use the "normalphoto" option if you want a normal photo instead of cropped to a circle
% \documentclass[10pt,a4paper,normalphoto]{altacv}

\documentclass[10pt,a4paper,ragged2e,withhyper]{altacv}

%% AltaCV uses the fontawesome5 and academicons fonts
%% and packages.
%% See http://texdoc.net/pkg/fontawesome5 and http://texdoc.net/pkg/academicons for full list of symbols. You MUST compile with XeLaTeX or LuaLaTeX if you want to use academicons.

% Change the page layout if you need to
\geometry{left=1.2cm,right=1.2cm,top=1cm,bottom=1cm,columnsep=0.75cm}

% The paracol package lets you typeset columns of text in parallel
\usepackage{paracol}

% Change the font if you want to, depending on whether
% you're using pdflatex or xelatex/lualatex
\ifxetexorluatex
  % If using xelatex or lualatex:
  \setmainfont{Roboto Slab}
  \setsansfont{Lato}
  \renewcommand{\familydefault}{\sfdefault}
\else
  % If using pdflatex:
  \usepackage[rm]{roboto}
  \usepackage[defaultsans]{lato}
  % \usepackage{sourcesanspro}
  \renewcommand{\familydefault}{\sfdefault}
\fi

% ----- LIGHT MODE -----
\definecolor{SlateGrey}{HTML}{2E2E2E}
\definecolor{LightGrey}{HTML}{666666}
\definecolor{PrimaryColor}{HTML}{196619}
\definecolor{SecondaryColor}{HTML}{2eb82e}
\definecolor{ThirdColor}{HTML}{196619}
\definecolor{BackgroundColor}{HTML}{ffffff}
\colorlet{name}{PrimaryColor}
\colorlet{tagline}{PrimaryColor}
\colorlet{heading}{PrimaryColor}
\colorlet{headingrule}{ThirdColor}
\colorlet{subheading}{SecondaryColor}
\colorlet{accent}{SecondaryColor}
\colorlet{emphasis}{SlateGrey}
\colorlet{body}{LightGrey}
\pagecolor{BackgroundColor}   
% ----- DARK MODE -----
%\definecolor{BackgroundColor}{HTML}{242424}
%\definecolor{SlateGrey}{HTML}{6F6F6F}
%\definecolor{LightGrey}{HTML}{ABABAB}
%\definecolor{PrimaryColor}{HTML}{3F7FFF}
%\colorlet{name}{PrimaryColor}
%\colorlet{tagline}{PrimaryColor}
%\colorlet{heading}{PrimaryColor}
%\colorlet{headingrule}{PrimaryColor}
%\colorlet{subheading}{PrimaryColor}
%\colorlet{accent}{PrimaryColor}
%\colorlet{emphasis}{LightGrey}
%\colorlet{body}{LightGrey}
%\pagecolor{BackgroundColor}

% Change some fonts, if necessary
\renewcommand{\namefont}{\Huge\rmfamily\bfseries}
\renewcommand{\personalinfofont}{\small\bfseries}
\renewcommand{\cvsectionfont}{\LARGE\rmfamily\bfseries}
\renewcommand{\cvsubsectionfont}{\large\bfseries}

% Change the bullets for itemize and rating marker
% for \cvskill if you want to
\renewcommand{\itemmarker}{{\small\textbullet}}
\renewcommand{\ratingmarker}{\faCircle}

%% sample.bib contains your publications
%% \addbibresource{sample.bib}

\begin{document}
    \name{Murali Wood}
    \tagline{FPGA Developer}
    %% You can add multiple photos on the left or right
    \photoL{4cm}{Astro3}
    
    \personalinfo{
        \email{mugz.wood@gmail.com}\smallskip
        \phone{+44 (0)7----------5}
        \location{Bangor, Wales}\\
        \linkedin{murali-wood-46957a17a}
        \github{MogoBagginz}
        %\homepage{nicolasomar.me}
        %\medium{nicolasomar}
        %% You MUST add the academicons option to \documentclass, then compile with LuaLaTeX or XeLaTeX, if you want to use \orcid or other academicons commands.
        % \orcid{0000-0000-0000-0000}
        %% You can add your own arbtrary detail with
        %% \printinfo{symbol}{detail}[optional hyperlink prefix]
        % \printinfo{\faPaw}{Hey ho!}[https://example.com/]
        %% Or you can declare your own field with
        %% \NewInfoFiled{fieldname}{symbol}[optional hyperlink prefix] and use it:
        % \NewInfoField{gitlab}{\faGitlab}[https://gitlab.com/]
        % \gitlab{your_id}
    }
    
    \makecvheader
    %% Depending on your tastes, you may want to make fonts of itemize environments slightly smaller
    % \AtBeginEnvironment{itemize}{\small}
    
    %% Set the left/right column width ratio to 6:4.
    \columnratio{0.25}

    % Start a 2-column paracol. Both the left and right columns will automatically
    % break across pages if things get too long.
    \begin{paracol}{2}
    
         % ----- ABOUT ME -----
        \cvsection{About Me}
            \begin{quote}
                Space enthusiast, keen electronics hobbyist, technically capable, creative thinker who enjoys exploring the limits of what is possible. 
            \end{quote}
        % ----- ABOUT ME -----
        
        % ----- STRENGTHS -----
        \cvsection{Strengths}
            \cvtag{C}
            \cvtag{Verilog}
            \cvtag{Digital design}
            \cvtag{Apio}
            \cvtag{Quartus}
            \cvtag{Soldering}
            \cvtag{Java}
            \cvtag{Pro Tools}
            \cvtag{easyEDA}
            \cvtag{Arduino}
            \cvtag{Photoshop}
            \cvtag{Illustrator}
            \medskip
        % ----- STRENGTHS -----
        
        % ----- LEARNING -----
        \cvsection{Learning}
            \cvtag{VHDL}
            \cvtag{DaVinci Resolve}\\
            \cvtag{Fusion360}
            \cvtag{3D printing}\\
            \cvtag{Drone Building}
            \cvtag{Latex}
            \cvtag{MatLab}
            \medskip
        % ----- LEARNING -----
        
         % ----- Courses -----
        \cvsection{Courses}
            \cvlang{Environmental First Aid}{Outreach Rescue Medical Skills}
            \smallskip
        % ----- Courses -----
        
        % ----- LANGUAGES -----
        \cvsection{Languages}
            \cvlang{English}{Native}\\
            \cvlang{German}{Basic}\\
            \cvlang{Welsh}{Basic}
            %% Yeah I didn't spend too much time making all the
            %% spacing consistent... sorry. Use \smallskip, \medskip,
            %% \bigskip, \vpsace etc to make ajustments.
            \smallskip
        % ----- LANGUAGES -----
                   
        % ----- REFERENCES -----
        \cvsection{References}
            \cvlang{Dr Roger Giddings\\ (Project Supervisor)}{}\\
            \email{r.p.giddings@bangor.ac.uk}
            \smallskip
        % ----- REFERENCES -----
        
        % ----- MOST PROUD -----
        % \cvsection{Most Proud of}
        
        % \cvachievement{\faTrophy}{Fantastic Achievement}{and some details about it}\\
        % \divider
        % \cvachievement{\faHeartbeat}{Another achievement}{more details about it of course}\\
        % \divider
        % \cvachievement{\faHeartbeat}{Another achievement}{more details about it of course}
        % ----- MOST PROUD -----
        
        % \cvsection{A Day of My Life}
        
        % Adapted from @Jake's answer from http://tex.stackexchange.com/a/82729/226
        % \wheelchart{outer radius}{inner radius}{
        % comma-separated list of value/text width/color/detail}
        % \wheelchart{1.5cm}{0.5cm}{%
        %   6/8em/accent!30/{Sleep,\\beautiful sleep},
        %   3/8em/accent!40/Hopeful novelist by night,
        %   8/8em/accent!60/Daytime job,
        %   2/10em/accent/Sports and relaxation,
        %   5/6em/accent!20/Spending time with family
        % }
        
        % use ONLY \newpage if you want to force a page break for
        % ONLY the current column
        \newpage
        
        %% Switch to the right column. This will now automatically move to the second
        %% page if the content is too long.
        \switchcolumn

        % ----- EDUCATION -----
        \cvsection{Education}
            \cvevent{BEng Electronic Engineering }{| Bangor University}{Sept 2018 -- June 2021}{Bangor, Wales}
            \begin{quote}
                Relevant modules - 	Digital Circuits Design 1 (80\%), Imperative Programming in C (98\%), VLSI Design Principles (81\%), Optoelectronics (71\%), Digital Circuits 2 (75\%), Communication Systems (77\%), Project, Planning & Management (87\%), Circuit Design (83\%), Digital Circuits & Design 1 (80\%), Optical Communication,  Information & Coding - Comms,  Signal Processing.
            \end{quote}
            \smallskip
            \cvtag{Predicted grade: First}

        % ----- EDUCATION -----
        
        % ----- PROJECTS -----
        \cvsection{Projects}
            \cvevent{Laser-Based Free Space Comunications}{\cvrepo{| \faGithub}{https://github.com/MogoBagginz/Laser-Based-FSO-TinyFPGA-BX}}{}{}{}
            \begin{quote}
                The project involved designing and building a transmitter and receiver, writing the Verilog from scratch including a TestBench file, PRBS generation using LFSR and a BER detector. The FPGA was a tinyFPGA BX which uses the ICE40PL8k chip from lattice. It achieved a bandwidth of 2.6 Mbps when using a 5 mW laser.
            \end{quote}
            \bigskip
            %\divider
            
            %\cvevent{FPGA Calculator }{\cvrepo{| %\faGithub}{https://github.com/MogoBagginz/TinyFPGACalculator}{}}{}{}
            %\begin{quote}
             %   As a way to learn a HDL I designed a calculator that accepted 8-bit numbers. The number %is stored in a register and could be added, subtracted, multiplied or divided by another %8-bit number. The sum is displayed using a 4x seven segment display.
            %\end{quote}
        % ----- PROJECTS -----
                
        % ----- EXPERIENCE -----
        \cvsection{Experience}
            \cvevent{President }{| MakerSpace Society}{2018 -- 2021}{Bangor University}            \begin{quote}
                The MakerSpace is a student run society that organises weekly meet ups and special events such as an amateur robot building competition Hebocon. We have a workshop in the university that we keep stocked and maintained. 
            \end{quote}
  
            \bigskip
            \cvevent{Band member/manager, Street musician }{| Musician}{2019 -- 2016}{25+ countries}
            \begin{quote}
                By working on recording a demo, creating an online presence and organising events I took my band The Rude 'em Outs from playing pub gigs to headlining festivals in two years. I have travelled extensively throughout Europe, Asia and Australia, supporting myself as a musician.
            \end{quote}
            
            
            %\divider
            \bigskip
            
            \cvevent{Volunteer }{| North Wales Tech}{2020}{Bangor, Wales}
            \begin{itemize}
                \item Helped out organising Tech/Maker events 
            \end{itemize}
        % ----- EXPERIENCE -----
        
    \end{paracol}
\end{document}
